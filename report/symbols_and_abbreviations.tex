\section*{Обозначения и сокращения}
\addcontentsline{toc}{section}{Обозначения и сокращения}
Используемые далее по тексту общие обозначения:
\newcommand*{\ditem}[1]{\item[#1~---]}
\begin{itemize}
\ditem{СК} система координат;
\ditem{КП} кинематическая пара;
\ditem{ДХ} Денавита-Хартенберга (Denavit–Hartenberg), например, соглашение;
\ditem{$q_i$} $i$-ая ($i=\overline{1,n}$) обобщенная координата манипулятора (угол, регистрируемый энкодером робота в $i$-ом сочленении);
\ditem{$n$} количество звеньев робота, $n = 5$;
\ditem{${}^iR_j$} матрица поворота, характеризующая поворот СК $0x_{j}y_{j}z_{j}$ относительно СК $0x_{i}y_{i}z_{i}$;
\ditem{${}^iA_j$} матрица однородных преобразований, описывающая смещение и поворот СК $0x_{j}y_{j}z_{j}$ относительно СК $0x_{i}y_{i}z_{i}$
\ditem{$r^i_{j,\,k}$} вектор из начала $0x_{j}y_{j}z_{j}$ в начало $0x_{k}y_{k}z_{k}$, выраженный относительно  $0x_{i}y_{i}z_{i}$;
\ditem{$g_i$} ускорение свободного падения, выраженное относительно $0x_{i}y_{i}z_{i}$;
\ditem{$V^i_j$} линейная скорость начала $0x_{j}y_{j}z_{j}$ относительно используемой в решении инерциальной системы отсчета (ИСО), выраженная относительно $0x_{i}y_{i}z_{i}$;
\ditem{$a^i_j$} линейное ускорение начала $0x_{j}y_{j}z_{j}$ относительно ИСО, выраженное относительно $0x_{i}y_{i}z_{i}$;
\ditem{$\omega^i_j$} угловая скорость вращения $0x_{j}y_{j}z_{j}$ относительно ИСО, выраженная относительно $0x_{i}y_{i}z_{i}$;
\ditem{$\omega^i_{j,\,k}$} угловая скорость вращения $0x_{k}y_{k}z_{k}$ относительно $0x_{j}y_{j}z_{j}$, выраженная относительно $0x_{i}y_{i}z_{i}$;
\ditem{$\dot\omega^i_j$} угловое ускорение $0x_{j}y_{j}z_{j}$ относительно ИСО, выраженное относительно $0x_{i}y_{i}z_{i}$;
\ditem{$z^i_j$} орт $[0\;0\;1]^T$ системы координат $0x_{j}y_{j}z_{j}$, выраженный относительно $0x_{i}y_{i}z_{i}$;
\ditem{$f^i_j$} сила, действующая на $j$-ое звено (тело) механизма со стороны $(j-1)$-го звена (тела), выраженная относительно $0x_{i}y_{i}z_{i}$;
\ditem{$\tau^i_j$} момент силы, действующий на $j$-ое звено (тело) механизма со стороны ${(j-1)}$-го звена (тела), выраженный относительно $0x_{i}y_{i}z_{i}$;
\ditem{$\tau_i$} обобщенный момент, ответственный за изменение обобщенной координаты $q_i$;
\ditem{$m_i$} масса $i$-го звена;
\ditem{$\mathcal{I}^i_j$} тензор инерции $j$-го звена, выраженный относительно жестко связанной с его центром масс системой координат, сонаправленной с $0x_{i}y_{i}z_{i}$;
\ditem{$a_i, d_i$} обозначения для длин, входящих в число параметров Де\-на\-ви\-та-Хар\-тен\-бер\-га, $i=\overline{1,n}$;
\ditem{$\alpha_i, \theta_i$} обозначения для углов, входящих в число параметров Де\-на\-ви\-та-Хар\-тен\-бер\-га, $i=\overline{1,n}$;
\end{itemize}
\newpage
